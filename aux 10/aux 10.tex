\documentclass[letterpaper,11pt]{article}
\oddsidemargin -1.0cm \textwidth 17.4cm

\usepackage[utf8]{inputenc}
\usepackage[activeacute,spanish]{babel}
\usepackage{amsfonts,setspace}
\usepackage{amsmath}
\usepackage{amssymb, amsmath, amsthm}
\usepackage{comment}
\usepackage{amssymb}
\usepackage{dsfont}
\usepackage{anysize}
\usepackage{multicol}
\usepackage{enumerate}
\usepackage{graphicx}
\usepackage[left=2cm,top=2cm,right=2cm, bottom=2cm]{geometry}
\setlength\headheight{2em} 
\usepackage{fancyhdr}
\usepackage{multicol}
\pagestyle{fancy}
\fancyhf{}


\renewcommand{\labelenumi}{\normalsize\bfseries P\arabic{enumi}.}
\renewcommand{\labelenumii}{\normalsize\bfseries (\alph{enumii})}
\renewcommand{\labelenumiii}{\normalsize\bfseries \roman{enumiii})}


\DeclareMathOperator{\sen}{sen}
\DeclareMathOperator{\senh}{senh}
\DeclareMathOperator{\arcsen}{arcsen}
\DeclareMathOperator{\tg}{tg}
\DeclareMathOperator{\arctg}{arctg}
\DeclareMathOperator{\ctg}{ctg}
\DeclareMathOperator{\dom}{Dom}
\DeclareMathOperator{\sech}{sech}
\DeclareMathOperator{\rec}{Rec}
\DeclareMathOperator{\inte}{Int}
\DeclareMathOperator{\adh}{Adh}
\DeclareMathOperator{\fr}{Fr}
\DeclareMathOperator{\Ima}{Im}
\DeclareMathOperator{\dist}{dist}
\DeclareMathOperator{\argmin}{\text{argmín}}
\let\lim=\undefined\DeclareMathOperator*{\lim}{\text{lím}}
\let\max=\undefined\DeclareMathOperator*{\max}{\text{máx}}
\let\min=\undefined\DeclareMathOperator*{\min}{\text{mín}}
\let\inf=\undefined\DeclareMathOperator*{\inf}{\text{ínf}}


\newcommand{\pint}[2]{\left< #1,#2\right>}
\newcommand{\ssi}{\Longleftrightarrow}
\newcommand{\conv}[2]{\xrightarrow[#1\to#2]{}}
\newcommand{\imp}{\Longrightarrow}
\newcommand{\pmi}{\Longleftarrow}
\newcommand{\ipartial}[2]{\dfrac{\partial #1}{\partial #2}}
\newcommand{\ider}[2]{\dfrac{d #1}{d #2}}
\newcommand{\iipartial}[2]{\dfrac{\partial^2 #1}{\partial #2^2}}
\newcommand{\iider}[2]{\dfrac{d^2 #1}{d #2^2}}
\newcommand{\ijpartial}[3]{\dfrac{\partial^2 #1}{\partial #2 \partial #3}}
\newcommand{\N}{\mathbb{N}}
\newcommand{\Z}{\mathbb{Z}}
\newcommand{\C}{\mathbb{C}}
\newcommand{\Q}{\mathbb{Q}}
\newcommand{\R}{\mathbb{R}}
\newcommand{\K}{\mathbb{K}}
\newcommand{\sol}{\textbf{\emph{Soluci\'on: }}}
\newcommand{\dem}{\textbf{\emph{Demostraci\'on: }}}
\newcommand{\aux}[4]{\Large \textbf{Clase Auxiliar N#1: #2}}
\newcommand{\pauta}[4]{\Large \textbf{Pauta #1 N#2}}
\newcommand{\enc}[3]{\Large \textbf{#1}}
\newcommand{\norm}[1]{\lVert #1\rVert }
\newcommand{\vabs}[1]{\lvert #1\rvert}

\begin{document}

\fancyhead[L]{\itshape{Facultad de Ciencias F\'isicas y Matem\'aticas}}
\fancyhead[R]{\itshape{Universidad de Chile}}

\begin{minipage}{11.5 cm}
\begin{flushleft}
\hspace*{-0.6cm}\textbf{MA2001-4 Cálculo en Varias Variables}\\
\hspace*{-0.6cm}\textbf{Profesor:} Javier Ramírez G.\\
\hspace*{-0.6cm}\textbf{Auxiliar:} Alejandro Silva C.\\

\end{flushleft}
\end{minipage}

\begin{picture}(2,3)
    \put(370,-4){\includegraphics[scale=1.2]{fcfm2.pdf}}
\end{picture}

\begin{center}
	\LARGE \bf{Auxiliar \#10 }\\
\end{center}

\vspace{-1cm}
\begin{enumerate}\setlength{\itemsep}{0.4cm}	
\item[]

\item 

\begin{enumerate}
    \item Sean las funciones $f:\mathbb{R}^2\rightarrow\mathbb{R}$ y $g,h: \mathbb{R}\rightarrow\mathbb{R}$ todas de clase $\mathcal{C}^4$, se define la función $$F(x,y,z)=~f(g(x+y),h(y+z))$$ 
    Determine $\dfrac{\partial F }{\partial x}\text{, } \dfrac{\partial^2 F}{\partial x\partial y}\text{, } \dfrac{\partial^2 F}{\partial y \partial x}\text{ y } \dfrac{\partial^3F}{\partial x\partial y\partial z}$
    
    \item Sea $f:\Omega\subseteq\mathbb{R}^n\rightarrow\mathbb{R}$, se define la matriz Hessiana de $f$ en $x_0\in\Omega$ como:
    \[H_f(x_0)=D(\nabla f) (x_0)\]
    Consideremos la función $\displaystyle f(x,y)=xe^{xy^2}$, determine la matriz Hessiana asociada a esta función
    
    \item Sea $f:\mathbb{R}^2\rightarrow\mathbb{R}$ de clase $\mathcal{C}^2$. Decimos que $f(x,t)$ satisface la Ecuación de Onda si se tiene que:
    \[\dfrac{\partial^2f}{\partial x^2}-\dfrac{1}{c^2}\dfrac{\partial^2f}{\partial t^2}=0\]
    Considere $\phi:\mathbb{R}^2\rightarrow\mathbb{R}$ tal que $\phi(u(x,t),v(x,t))=\phi(x+ct,x-ct)=f(x,t)$.\par
    Muestre que si $f$ satisface la ecuación de onda, entonces
    \[\dfrac{\partial^2\phi}{\partial u\partial v}=0\]
\end{enumerate}

\item 
\begin{enumerate}
    \item
    \begin{itemize}
        \item[i.] Determine el plano tangente a la superficie definida por $z=x^3+y^3-6xy$ en el punto $(1,2,-3)$
        \item[ii.] 
        Sea $\Omega=\{(x,y)\in\mathbb{R}^2: x>0,y>0\}$. Sea además $f:\Omega\rightarrow\mathbb{R}$ la función definida por:
        \[f(x,y)=\sqrt{x^2+y^2}(x^2-y^2)\]
        Demuestre que $f$ es diferenciable en $\Omega$ y encuentre la ecuación del plano tangente al grafo de $f$ para todos los puntos $(x,y)\in\Omega$ tales que x=y.
    \end{itemize}
    
    \item
    \begin{itemize}
        \item[i.] Encuentre el plano tangente a la superficie definida por la ecuación
        \[z^3y+e^{xy-1}+x^5+x^2yz+z^6=5\]
        en el punto $(1,1,1)$
    
        \item[ii.] Sea $f:\mathbb{R}^3\rightarrow\mathbb{R}$ la función dada por:
        \[f(x,y,z)=ax^2y+by^2x+cz^2x\]
        Encuentre la ecuación del plano tangente a la superficie
        \[S=\{(x,y,z): f(x,y,z)=a+b+c\}\]
        en el punto $(1,1,1)$
    \end{itemize}
\end{enumerate}

\end{enumerate}
\end{document}