\documentclass[letterpaper,11pt]{article}
\oddsidemargin -1.0cm \textwidth 17.4cm

\usepackage[utf8]{inputenc}
\usepackage[activeacute,spanish]{babel}
\usepackage{amsfonts,setspace}
\usepackage{amsmath}
\usepackage{amssymb, amsmath, amsthm}
\usepackage{comment}
\usepackage{amssymb}
\usepackage{dsfont}
\usepackage{anysize}
\usepackage{multicol}
\usepackage{enumerate}
\usepackage{graphicx}
\usepackage[left=2cm,top=2cm,right=2cm, bottom=2cm]{geometry}
\setlength\headheight{2em} 
\usepackage{fancyhdr}
\usepackage{multicol}
\pagestyle{fancy}
\fancyhf{}


\renewcommand{\labelenumi}{\normalsize\bfseries P\arabic{enumi}.}
\renewcommand{\labelenumii}{\normalsize\bfseries (\alph{enumii})}
\renewcommand{\labelenumiii}{\normalsize\bfseries \roman{enumiii})}


\DeclareMathOperator{\sen}{sen}
\DeclareMathOperator{\senh}{senh}
\DeclareMathOperator{\arcsen}{arcsen}
\DeclareMathOperator{\tg}{tg}
\DeclareMathOperator{\arctg}{arctg}
\DeclareMathOperator{\ctg}{ctg}
\DeclareMathOperator{\dom}{Dom}
\DeclareMathOperator{\sech}{sech}
\DeclareMathOperator{\rec}{Rec}
\DeclareMathOperator{\inte}{Int}
\DeclareMathOperator{\adh}{Adh}
\DeclareMathOperator{\fr}{Fr}
\DeclareMathOperator{\Ima}{Im}
\DeclareMathOperator{\dist}{dist}
\DeclareMathOperator{\argmin}{\text{argmín}}
\let\lim=\undefined\DeclareMathOperator*{\lim}{\text{lím}}
\let\max=\undefined\DeclareMathOperator*{\max}{\text{máx}}
\let\min=\undefined\DeclareMathOperator*{\min}{\text{mín}}
\let\inf=\undefined\DeclareMathOperator*{\inf}{\text{ínf}}


\newcommand{\pint}[2]{\left< #1,#2\right>}
\newcommand{\ssi}{\Longleftrightarrow}
\newcommand{\conv}[2]{\xrightarrow[#1\to#2]{}}
\newcommand{\imp}{\Longrightarrow}
\newcommand{\pmi}{\Longleftarrow}
\newcommand{\ipartial}[2]{\dfrac{\partial #1}{\partial #2}}
\newcommand{\ider}[2]{\dfrac{d #1}{d #2}}
\newcommand{\iipartial}[2]{\dfrac{\partial^2 #1}{\partial #2^2}}
\newcommand{\iider}[2]{\dfrac{d^2 #1}{d #2^2}}
\newcommand{\ijpartial}[3]{\dfrac{\partial^2 #1}{\partial #2 \partial #3}}
\newcommand{\N}{\mathbb{N}}
\newcommand{\Z}{\mathbb{Z}}
\newcommand{\C}{\mathbb{C}}
\newcommand{\Q}{\mathbb{Q}}
\newcommand{\R}{\mathbb{R}}
\newcommand{\K}{\mathbb{K}}
\newcommand{\sol}{\textbf{\emph{Soluci\'on: }}}
\newcommand{\dem}{\textbf{\emph{Demostraci\'on: }}}
\newcommand{\aux}[4]{\Large \textbf{Clase Auxiliar N#1: #2}}
\newcommand{\pauta}[4]{\Large \textbf{Pauta #1 N#2}}
\newcommand{\enc}[3]{\Large \textbf{#1}}
\newcommand{\norm}[1]{\lVert #1\rVert }
\newcommand{\vabs}[1]{\lvert #1\rvert}

\begin{document}

\fancyhead[L]{\itshape{Facultad de Ciencias F\'isicas y Matem\'aticas}}
\fancyhead[R]{\itshape{Universidad de Chile}}

\begin{minipage}{11.5 cm}
\begin{flushleft}
\hspace*{-0.6cm}\textbf{MA2001-4 Cálculo en Varias Variables}\\
\hspace*{-0.6cm}\textbf{Profesor:} Javier Ramírez G.\\
\hspace*{-0.6cm}\textbf{Auxiliar:} Alejandro Silva C.\\

\end{flushleft}
\end{minipage}

\begin{picture}(2,3)
    \put(370,-4){\includegraphics[scale=1.2]{fcfm2.pdf}}
\end{picture}

\begin{center}
	\LARGE \bf{Auxiliar \#7 }\\
\end{center}

\vspace{-1cm}
\begin{enumerate}\setlength{\itemsep}{0.4cm}	
\item[]

\item
\begin{enumerate}
    \item Pruebe que el siguiente conjunto es compacto:
\[A=\left\{(x,y,z)\in\mathbb{R}^3\text{   }\bigg{|}\text{ } \frac{x^2}{2}+\frac{y^2}{4}+\frac{z^2}{8}\leq 1\right\}\]

\item Sea $f:\mathbb{R}^2\rightarrow\mathbb{R}$ definida por:
\begin{align*}
    \displaystyle
    f(x,y)=\left\{\begin{array}{cl}
    \frac{x^3-y^3}{xy}& \text{ si }xy\neq 0  \\
    0 &\text{ si }xy=0 
    \end{array}\right.
\end{align*}
Pruebe que el conjunto de los puntos de discontinuidad de $f$ es cerrado y determínelo
\end{enumerate}
\item Considere la función $f:\mathbb{R}^2\rightarrow\mathbb{R}$ definida por:
\[f(x,y)=(1-x^2-y^2)e^{5x^2-y^2+cos(3x+y)}\]
Sea $D=\left\{(x,y)\text{  }\big{|}\text{ } x^2+y^2<1\right\}$. Demuestre que existe un punto $(\bar{x},\bar{y})\in D$ tal que:
\[f(\bar{x},\bar{y})\geq f(x,y) \text{   } \forall (x,y)\in D \]
\item 
\begin{enumerate}
    \item Demuestre que toda norma es Lipschitz
    \item Demuestre que la función $f:\mathbb{R}_+\rightarrow\mathbb{R}$ definida por $f(x)=\sqrt{x}$ no es Lipschitz
\end{enumerate}

\item Sea $A\subseteq\mathbb{R}^n$ un conjunto compacto, y sea $x_0\notin A$ un punto fijo. Se define $f: A\rightarrow R$ como:
\[f(x)=\norm{x-x_0}\]
    \begin{enumerate}
        \item Demuestre que $f$ es una función continua
        \item Pruebe que $f$ alcanza máximo y mínimo sobre $A$
        \item Muestre que $\underset{x\in A}{\min}f(x)>0$
        \item Dada una constante $c>0$, tal que $\underset{x\in A}{\min}f(x)<c<\underset{x\in A}{\max}f(x)$, consideremos el conjunto
        \[A(c)=\left\{x\in A \text{ : } f(x)<c\right\}\]
        ¿Es $A(c)$ un conjunto abierto?
    \end{enumerate}
\end{enumerate}
\end{document}