\documentclass[letterpaper,11pt]{article}
\oddsidemargin -1.0cm \textwidth 17.4cm

\usepackage[utf8]{inputenc}
\usepackage[activeacute,spanish]{babel}
\usepackage{amsfonts,setspace}
\usepackage{amsmath}
\usepackage{amssymb, amsmath, amsthm}
\usepackage{comment}
\usepackage{amssymb}
\usepackage{dsfont}
\usepackage{anysize}
\usepackage{multicol}
\usepackage{enumerate}
\usepackage{graphicx}
\usepackage[left=2cm,top=2cm,right=2cm, bottom=2cm]{geometry}
\setlength\headheight{2em} 
\usepackage{fancyhdr}
\usepackage{multicol}
\pagestyle{fancy}
\fancyhf{}


\renewcommand{\labelenumi}{\normalsize\bfseries P\arabic{enumi}.}
\renewcommand{\labelenumii}{\normalsize\bfseries (\alph{enumii})}
\renewcommand{\labelenumiii}{\normalsize\bfseries \roman{enumiii})}


\DeclareMathOperator{\sen}{sen}
\DeclareMathOperator{\senh}{senh}
\DeclareMathOperator{\arcsen}{arcsen}
\DeclareMathOperator{\tg}{tg}
\DeclareMathOperator{\arctg}{arctg}
\DeclareMathOperator{\ctg}{ctg}
\DeclareMathOperator{\dom}{Dom}
\DeclareMathOperator{\sech}{sech}
\DeclareMathOperator{\rec}{Rec}
\DeclareMathOperator{\inte}{Int}
\DeclareMathOperator{\adh}{Adh}
\DeclareMathOperator{\fr}{Fr}
\DeclareMathOperator{\Ima}{Im}
\DeclareMathOperator{\dist}{dist}
\DeclareMathOperator{\argmin}{\text{argmín}}
\let\lim=\undefined\DeclareMathOperator*{\lim}{\text{lím}}
\let\max=\undefined\DeclareMathOperator*{\max}{\text{máx}}
\let\min=\undefined\DeclareMathOperator*{\min}{\text{mín}}
\let\inf=\undefined\DeclareMathOperator*{\inf}{\text{ínf}}


\newcommand{\pint}[2]{\left< #1,#2\right>}
\newcommand{\ssi}{\Longleftrightarrow}
\newcommand{\conv}[2]{\xrightarrow[#1\to#2]{}}
\newcommand{\imp}{\Longrightarrow}
\newcommand{\pmi}{\Longleftarrow}
\newcommand{\ipartial}[2]{\dfrac{\partial #1}{\partial #2}}
\newcommand{\ider}[2]{\dfrac{d #1}{d #2}}
\newcommand{\iipartial}[2]{\dfrac{\partial^2 #1}{\partial #2^2}}
\newcommand{\iider}[2]{\dfrac{d^2 #1}{d #2^2}}
\newcommand{\ijpartial}[3]{\dfrac{\partial^2 #1}{\partial #2 \partial #3}}
\newcommand{\N}{\mathbb{N}}
\newcommand{\Z}{\mathbb{Z}}
\newcommand{\C}{\mathbb{C}}
\newcommand{\Q}{\mathbb{Q}}
\newcommand{\R}{\mathbb{R}}
\newcommand{\K}{\mathbb{K}}
\newcommand{\sol}{\textbf{\emph{Soluci\'on: }}}
\newcommand{\dem}{\textbf{\emph{Demostraci\'on: }}}
\newcommand{\aux}[4]{\Large \textbf{Clase Auxiliar N#1: #2}}
\newcommand{\pauta}[4]{\Large \textbf{Pauta #1 N#2}}
\newcommand{\enc}[3]{\Large \textbf{#1}}
\newcommand{\norm}[1]{\lVert #1\rVert }
\newcommand{\vabs}[1]{\lvert #1\rvert}

\begin{document}

\fancyhead[L]{\itshape{Facultad de Ciencias F\'isicas y Matem\'aticas}}
\fancyhead[R]{\itshape{Universidad de Chile}}

\begin{minipage}{11.5 cm}
\begin{flushleft}
\hspace*{-0.6cm}\textbf{MA2001-4 Cálculo en Varias Variables}\\
\hspace*{-0.6cm}\textbf{Profesor:} Javier Ramírez G.\\
\hspace*{-0.6cm}\textbf{Auxiliar:} Alejandro Silva C.\\

\end{flushleft}
\end{minipage}

\begin{picture}(2,3)
    \put(370,-4){\includegraphics[scale=1.2]{fcfm2.pdf}}
\end{picture}

\begin{center}
	\LARGE \bf{Auxiliar \#5}\\
\end{center}

\vspace{-1cm}
\begin{enumerate}\setlength{\itemsep}{0.4cm}	
\item[]

\item En esta pregunta probaremos el siguiente resultado de gran utilidad:\\

\textit{\underline{\textbf{Prop:}} Sea $\Omega \subseteq E, f:E\to F$ y $x_0 \in \text{acc}(\Omega)\cap \Omega$, entonces
\[f \text{ es continua en } x_0 \ssi \forall\,(x_n)_{n\in\N}\subseteq\Omega\setminus\{x_0\}, x_n\conv{n}{\infty}x_0 \imp f(x_n)\conv{n}{\infty}f(x_0).\]
}
Esto quiere decir que para probar la continuidad de una función en un punto de acumulación, no es necesario considerar los casos en que la sucesión pasa por el punto.

\item Sea $f:\mathbb{R}^n\rightarrow\mathbb{R}^m$ una función continua en todo punto de $\mathbb{R}^n$. Se denota la imagen de un conjunto $A\subseteq\mathbb{R}^n$ por $f(A)=\{f(x)\in\mathbb{R}^m\text{ }|\text{ } x\in A\}$. Demuestre que para todo conjunto $A\subseteq\mathbb{R}^n$ se tiene que: 
\[f(\adh{(A)})\subseteq\adh{(f(A))}\]
Muestre que en general no se tiene la igualdad dando un ejemplo

\item Estudie los siguientes límites:
    \begin{enumerate}
        \begin{multicols}{3} 
            \item $\displaystyle \underset{(x,y)\rightarrow(0,0)}{\lim} x^4\sin{\left(\frac{1}{x^2+|y|}\right)}$
            \item $\displaystyle \underset{(x,y)\rightarrow(0,0)}{\lim} \frac{x^2}{x^2+y^2}$
            \item $\displaystyle \underset{(x,y)\rightarrow(0,0)}{\lim} \frac{xy\cos{(x)}}{x^2+y^2}$ 
        \end{multicols}
        \begin{multicols}{3}
            \item $\displaystyle \underset{(x,y,z)\rightarrow(0,0,0)}{\lim} \frac{x^2yz}{x^4+y^2x^2+z^4}$
            \item $\displaystyle \underset{(x,y)\rightarrow(0,0)}{\lim} \frac{xy\sin{(x)}}{x^2+y^2} $
        \end{multicols}
        
    \end{enumerate}
\item 
    \begin{enumerate}
        \item Estudie continuidad de la función $f: \mathbb{R}^2\rightarrow\mathbb{R}$ dada por:
        \begin{align*}
            f(x,y)=\left\{
            \begin{array}{cl}
            \displaystyle\frac{x\cos{y}-y\cos{x}-x+y}{x^2+y^2}& \text{ si } (x,y)\neq(0,0) \\
            0 & \text{ si } (x,y)=(0,0)
            \end{array} \right.
        \end{align*}
        
        \item Determine los valores de $\alpha$ para que la función $g : \mathbb{R}^2\rightarrow\mathbb{R}$ dada por:
            \begin{align*}
                g(x,y)=\left\{
                \begin{array}{cl}
                \displaystyle \frac{|xy|^{\alpha}}{x^2-xy+y^2}&  \text{ si } (x,y)\neq(0,0) \\
                0 &  \text{ si } (x,y)=(0,0)
                \end{array}
                \right.
            \end{align*}
            Sea continua
        \item Considere la función $h: \mathbb{R}^2\rightarrow\mathbb{R}$ dada por:
        \begin{align*}
            h(x,y)=\left\{
            \begin{array}{cl}
            x+y &\text{ si } x+y\leq0  \\
            \sqrt{x+y}+xy& x+y>0
            \end{array}\right.
        \end{align*}
        Determine los puntos de $\mathbb{R}^2$ donde $h$ es continua
        
    \end{enumerate}
\end{enumerate}
\end{document}