\documentclass[letterpaper,11pt]{article}
\oddsidemargin -1.0cm \textwidth 17.4cm

\usepackage[utf8]{inputenc}
\usepackage[activeacute,spanish]{babel}
\usepackage{amsfonts,setspace}
\usepackage{amsmath}
\usepackage{amssymb, amsmath, amsthm}
\usepackage{comment}
\usepackage{amssymb}
\usepackage{dsfont}
\usepackage{anysize}
\usepackage{multicol}
\usepackage{enumerate}
\usepackage{graphicx}
\usepackage[left=2cm,top=2cm,right=2cm, bottom=2cm]{geometry}
\setlength\headheight{2em} 
\usepackage{fancyhdr}
\usepackage{multicol}
\pagestyle{fancy}
\fancyhf{}


\renewcommand{\labelenumi}{\normalsize\bfseries P\arabic{enumi}.}
\renewcommand{\labelenumii}{\normalsize\bfseries (\alph{enumii})}
\renewcommand{\labelenumiii}{\normalsize\bfseries \roman{enumiii})}


\DeclareMathOperator{\sen}{sen}
\DeclareMathOperator{\senh}{senh}
\DeclareMathOperator{\arcsen}{arcsen}
\DeclareMathOperator{\tg}{tg}
\DeclareMathOperator{\arctg}{arctg}
\DeclareMathOperator{\ctg}{ctg}
\DeclareMathOperator{\dom}{Dom}
\DeclareMathOperator{\sech}{sech}
\DeclareMathOperator{\rec}{Rec}
\DeclareMathOperator{\inte}{Int}
\DeclareMathOperator{\adh}{Adh}
\DeclareMathOperator{\fr}{Fr}
\DeclareMathOperator{\Ima}{Im}
\DeclareMathOperator{\dist}{dist}
\DeclareMathOperator{\argmin}{\text{argmín}}
\let\lim=\undefined\DeclareMathOperator*{\lim}{\text{lím}}
\let\max=\undefined\DeclareMathOperator*{\max}{\text{máx}}
\let\min=\undefined\DeclareMathOperator*{\min}{\text{mín}}
\let\inf=\undefined\DeclareMathOperator*{\inf}{\text{ínf}}


\newcommand{\pint}[2]{\left< #1,#2\right>}
\newcommand{\ssi}{\Longleftrightarrow}
\newcommand{\conv}[2]{\xrightarrow[#1\to#2]{}}
\newcommand{\imp}{\Longrightarrow}
\newcommand{\pmi}{\Longleftarrow}
\newcommand{\ipartial}[2]{\dfrac{\partial #1}{\partial #2}}
\newcommand{\ider}[2]{\dfrac{d #1}{d #2}}
\newcommand{\iipartial}[2]{\dfrac{\partial^2 #1}{\partial #2^2}}
\newcommand{\iider}[2]{\dfrac{d^2 #1}{d #2^2}}
\newcommand{\ijpartial}[3]{\dfrac{\partial^2 #1}{\partial #2 \partial #3}}
\newcommand{\N}{\mathbb{N}}
\newcommand{\Z}{\mathbb{Z}}
\newcommand{\C}{\mathbb{C}}
\newcommand{\Q}{\mathbb{Q}}
\newcommand{\R}{\mathbb{R}}
\newcommand{\K}{\mathbb{K}}
\newcommand{\sol}{\textbf{\emph{Soluci\'on: }}}
\newcommand{\dem}{\textbf{\emph{Demostraci\'on: }}}
\newcommand{\aux}[4]{\Large \textbf{Clase Auxiliar N#1: #2}}
\newcommand{\pauta}[4]{\Large \textbf{Pauta #1 N#2}}
\newcommand{\enc}[3]{\Large \textbf{#1}}
\newcommand{\norm}[1]{\lVert #1\rVert }
\newcommand{\vabs}[1]{\lvert #1\rvert}

\begin{document}

\fancyhead[L]{\itshape{Facultad de Ciencias F\'isicas y Matem\'aticas}}
\fancyhead[R]{\itshape{Universidad de Chile}}

\begin{minipage}{11.5 cm}
\begin{flushleft}
\hspace*{-0.6cm}\textbf{MA2001-4 Cálculo en Varias Variables}\\
\hspace*{-0.6cm}\textbf{Profesor:} Javier Ramírez G.\\
\hspace*{-0.6cm}\textbf{Auxiliar:} Alejandro Silva C.\\

\end{flushleft}
\end{minipage}

\begin{picture}(2,3)
    \put(370,-4){\includegraphics[scale=1.2]{fcfm2.pdf}}
\end{picture}

\begin{center}
	\LARGE \bf{Auxiliar \#8 }\\
\end{center}

\vspace{-1cm}
\begin{enumerate}\setlength{\itemsep}{0.4cm}	
\item[]

\item
\begin{enumerate}
    \item Sea $f: \mathbb{R}^n\rightarrow\mathbb{R}$ dada por la forma cuadrática
    \[f(x)=x^\intercal Ax\]
    donde $A$ es una matriz de $n\times n$. \par
    Dado $x_0\in\mathbb{R}^n$ determine que $f$ es diferenciable en $x_0$ y encuentre $f^{\prime}(x_0)$
    
    \item Sea $g: \mathbb{R}^n\rightarrow\mathbb{R}$ definida por $g(x)=\norm{x}^2_2$. Determine la matriz $Dg(x)$
\end{enumerate}
\item Sea $g:\mathbb{R}^2\rightarrow\mathbb{R}^2$ definida por
\begin{align*}
    g(x,y)= 
    \begin{bmatrix}
    xy^3+x^2y\\
    y^2
    \end{bmatrix}
\end{align*}
Pruebe que la matriz 
\begin{align*}
    Dg(x,y)=
    \begin{bmatrix}
    y^3+2xy & 3xy^2+x^2\\
    0 & 2y
    \end{bmatrix}
\end{align*}
define el diferencial de $g$ en $(0,1)$, i.e, $Dg(0,1)=g^{\prime}(0,1)$.\par
\textit{Hint:} Considere la desigualdad $|hk|+h^2\leq2(h^2+k^2)$

\item Sea $f:\mathbb{R}^3\rightarrow\mathbb{R}^2$ una función diferenciable y $g:\mathbb{R}^2\rightarrow\mathbb{R}^2$ definida por:
\[g(u,v)=f\left(cos(u)+sin(v),sin(u)+cos(v), e^{u-v}\right)\]

\begin{enumerate}
    \item Note que $g(u,v)$ se puede expresar como $(f \circ h)(u,v)$. Determine la función $h$ y pruebe que la matriz
    \begin{align*}
        Dh(u,v)=\begin{bmatrix}
            -sin(u) & cos(v) \\
            cos(u) & -sin(v) \\
            e^{u-v} & -e^{u-v} 
        \end{bmatrix}
    \end{align*}
    define el diferencial de $h$ en $(\frac{\pi}{2},\frac{\pi}{2})$.
    \item Usando la fórmula del diferencial de la composición de dos funciones y sabiendo que la matriz asociada a $Df(1,1,1)$ es
    \begin{align*}
        \begin{bmatrix}
        1 & 3 & 4 \\
        2 & -1 & 3
        \end{bmatrix}
    \end{align*}
    Calcule la matriz asociada a $Dg(\frac{\pi}{2},\frac{\pi}{2})$
\end{enumerate}

\end{enumerate}
\end{document}