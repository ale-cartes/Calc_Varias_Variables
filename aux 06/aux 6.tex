\documentclass[letterpaper,11pt]{article}
\oddsidemargin -1.0cm \textwidth 17.4cm

\usepackage[utf8]{inputenc}
\usepackage[activeacute,spanish]{babel}
\usepackage{amsfonts,setspace}
\usepackage{amsmath}
\usepackage{amssymb, amsmath, amsthm}
\usepackage{comment}
\usepackage{amssymb}
\usepackage{dsfont}
\usepackage{anysize}
\usepackage{multicol}
\usepackage{enumerate}
\usepackage{graphicx}
\usepackage[left=2cm,top=2cm,right=2cm, bottom=2cm]{geometry}
\setlength\headheight{2em} 
\usepackage{fancyhdr}
\usepackage{multicol}
\pagestyle{fancy}
\fancyhf{}


\renewcommand{\labelenumi}{\normalsize\bfseries P\arabic{enumi}.}
\renewcommand{\labelenumii}{\normalsize\bfseries (\alph{enumii})}
\renewcommand{\labelenumiii}{\normalsize\bfseries \roman{enumiii})}


\DeclareMathOperator{\sen}{sen}
\DeclareMathOperator{\senh}{senh}
\DeclareMathOperator{\arcsen}{arcsen}
\DeclareMathOperator{\tg}{tg}
\DeclareMathOperator{\arctg}{arctg}
\DeclareMathOperator{\ctg}{ctg}
\DeclareMathOperator{\dom}{Dom}
\DeclareMathOperator{\sech}{sech}
\DeclareMathOperator{\rec}{Rec}
\DeclareMathOperator{\inte}{Int}
\DeclareMathOperator{\adh}{Adh}
\DeclareMathOperator{\fr}{Fr}
\DeclareMathOperator{\Ima}{Im}
\DeclareMathOperator{\dist}{dist}
\DeclareMathOperator{\argmin}{\text{argmín}}
\let\lim=\undefined\DeclareMathOperator*{\lim}{\text{lím}}
\let\max=\undefined\DeclareMathOperator*{\max}{\text{máx}}
\let\min=\undefined\DeclareMathOperator*{\min}{\text{mín}}
\let\inf=\undefined\DeclareMathOperator*{\inf}{\text{ínf}}


\newcommand{\pint}[2]{\left< #1,#2\right>}
\newcommand{\ssi}{\Longleftrightarrow}
\newcommand{\conv}[2]{\xrightarrow[#1\to#2]{}}
\newcommand{\imp}{\Longrightarrow}
\newcommand{\pmi}{\Longleftarrow}
\newcommand{\ipartial}[2]{\dfrac{\partial #1}{\partial #2}}
\newcommand{\ider}[2]{\dfrac{d #1}{d #2}}
\newcommand{\iipartial}[2]{\dfrac{\partial^2 #1}{\partial #2^2}}
\newcommand{\iider}[2]{\dfrac{d^2 #1}{d #2^2}}
\newcommand{\ijpartial}[3]{\dfrac{\partial^2 #1}{\partial #2 \partial #3}}
\newcommand{\N}{\mathbb{N}}
\newcommand{\Z}{\mathbb{Z}}
\newcommand{\C}{\mathbb{C}}
\newcommand{\Q}{\mathbb{Q}}
\newcommand{\R}{\mathbb{R}}
\newcommand{\K}{\mathbb{K}}
\newcommand{\sol}{\textbf{\emph{Soluci\'on: }}}
\newcommand{\dem}{\textbf{\emph{Demostraci\'on: }}}
\newcommand{\aux}[4]{\Large \textbf{Clase Auxiliar N#1: #2}}
\newcommand{\pauta}[4]{\Large \textbf{Pauta #1 N#2}}
\newcommand{\enc}[3]{\Large \textbf{#1}}
\newcommand{\norm}[1]{\lVert #1\rVert }
\newcommand{\vabs}[1]{\lvert #1\rvert}

\begin{document}

\fancyhead[L]{\itshape{Facultad de Ciencias F\'isicas y Matem\'aticas}}
\fancyhead[R]{\itshape{Universidad de Chile}}

\begin{minipage}{11.5 cm}
\begin{flushleft}
\hspace*{-0.6cm}\textbf{MA2001-4 Cálculo en Varias Variables}\\
\hspace*{-0.6cm}\textbf{Profesor:} Javier Ramírez G.\\
\hspace*{-0.6cm}\textbf{Auxiliar:} Alejandro Silva C.\\

\end{flushleft}
\end{minipage}

\begin{picture}(2,3)
    \put(370,-4){\includegraphics[scale=1.2]{fcfm2.pdf}}
\end{picture}

\begin{center}
	\LARGE \bf{Auxiliar \#6: Repaso C1}\\
\end{center}

\vspace{-1cm}
\begin{enumerate}\setlength{\itemsep}{0.4cm}	
\item[]

\item
    \begin{enumerate}
        \item[\textbf{(a.i)}] Dada una norma $\norm{\cdot}$ en $\mathbb{R}^n$ y una matriz $P$ no singular de $n\times n$. Demuestre que la función $N(x)=\norm{Px}$ es también una norma de $\mathbb{R}^n$
        \item[\textbf{(a.ii)}] Demuestre que en $\mathbb{R}^3$, la función $\displaystyle N_1(x,y,z)=\sqrt{\frac{x^2}{9}+4y^2+16z^2}$ es una norma.
        \item[\textbf{(b)}] Pruebe que 
        \[\norm{(x,y)}=\sqrt{x^2+xy+y^2}\]
        Define una norma en $\mathbb{R}^2$
    \end{enumerate}
\item Determine el interior, adherencia y frontera de los siguientes conjuntos:
    \begin{enumerate}
        \item $B=\left\{\left(\frac{1}{k},(-1)^k\right)\in\mathbb{R}^2 \text{ : } k\in\mathbb{N} \right\}$
        \item $C=\left\{ \left(\frac{1}{m},\frac{1}{n}\right) \text{  / } n,m\in\mathbb{N}\right\}$
    \end{enumerate}
\item Muestre que
    \begin{enumerate}
        \item A es cerrado si y solo si $\fr{(A)}\subseteq A$
        \item A es abierto si y solo si $A\cap\fr(A)=\emptyset$
    \end{enumerate}
\item 
    \begin{enumerate}
        \item Considere la función $f:\mathbb{R}^2\rightarrow\mathbb{R}$ definida por:
        \begin{align*}
            \displaystyle f(x,y)=\left\{
                \begin{array}{cl}
                    \frac{x^2y^2}{x^2y^2+(x-y)^2} &\text{ si } (x,y)\neq(0,0)  \\
                    1 &\text{ si } (x,y)=(0,0)
                \end{array}\right.
        \end{align*}
            Estudie la continuidad de $f$ en $\mathbb{R}^2$
        \item Considere la función $g:\left[0,\frac{\pi}{4}\right]\times[0,1]\rightarrow\mathbb{R}$ definida por:
        \begin{align*}
            \displaystyle g(x,y)=\left\{
                \begin{array}{cl}
                    tan(xy)^{\frac{1}{1-tan(xy)}}&\text{ si } (x,y)\neq\left(\frac{\pi}{4},1\right) \\
                    \gamma &\text{ si } (x,y)=\left(\frac{\pi}{4},1\right)
                \end{array}\right.
        \end{align*}
        Determine $\gamma$ de tal modo que $g$ sea continua en $\left[0,\frac{\pi}{4}\right]\times[0,1]$
    \end{enumerate}

\end{enumerate}
\end{document}