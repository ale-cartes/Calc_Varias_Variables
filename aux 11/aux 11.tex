\documentclass[letterpaper,11pt]{article}
\oddsidemargin -1.0cm \textwidth 17.4cm

\usepackage[utf8]{inputenc}
\usepackage[activeacute,spanish]{babel}
\usepackage{amsfonts,setspace}
\usepackage{amsmath}
\usepackage{amssymb, amsmath, amsthm}
\usepackage{comment}
\usepackage{amssymb}
\usepackage{dsfont}
\usepackage{anysize}
\usepackage{multicol}
\usepackage{enumerate}
\usepackage{graphicx}
\usepackage[left=2cm,top=2cm,right=2cm, bottom=2cm]{geometry}
\setlength\headheight{2em} 
\usepackage{fancyhdr}
\usepackage{multicol}
\pagestyle{fancy}
\fancyhf{}


\renewcommand{\labelenumi}{\normalsize\bfseries P\arabic{enumi}.}
\renewcommand{\labelenumii}{\normalsize\bfseries (\alph{enumii})}
\renewcommand{\labelenumiii}{\normalsize\bfseries \roman{enumiii})}


\DeclareMathOperator{\sen}{sen}
\DeclareMathOperator{\senh}{senh}
\DeclareMathOperator{\arcsen}{arcsen}
\DeclareMathOperator{\tg}{tg}
\DeclareMathOperator{\arctg}{arctg}
\DeclareMathOperator{\ctg}{ctg}
\DeclareMathOperator{\dom}{Dom}
\DeclareMathOperator{\sech}{sech}
\DeclareMathOperator{\rec}{Rec}
\DeclareMathOperator{\inte}{Int}
\DeclareMathOperator{\adh}{Adh}
\DeclareMathOperator{\fr}{Fr}
\DeclareMathOperator{\Ima}{Im}
\DeclareMathOperator{\dist}{dist}
\DeclareMathOperator{\argmin}{\text{argmín}}
\let\lim=\undefined\DeclareMathOperator*{\lim}{\text{lím}}
\let\max=\undefined\DeclareMathOperator*{\max}{\text{máx}}
\let\min=\undefined\DeclareMathOperator*{\min}{\text{mín}}
\let\inf=\undefined\DeclareMathOperator*{\inf}{\text{ínf}}


\newcommand{\pint}[2]{\left< #1,#2\right>}
\newcommand{\ssi}{\Longleftrightarrow}
\newcommand{\conv}[2]{\xrightarrow[#1\to#2]{}}
\newcommand{\imp}{\Longrightarrow}
\newcommand{\pmi}{\Longleftarrow}
\newcommand{\ipartial}[2]{\dfrac{\partial #1}{\partial #2}}
\newcommand{\ider}[2]{\dfrac{d #1}{d #2}}
\newcommand{\iipartial}[2]{\dfrac{\partial^2 #1}{\partial #2^2}}
\newcommand{\iider}[2]{\dfrac{d^2 #1}{d #2^2}}
\newcommand{\ijpartial}[3]{\dfrac{\partial^2 #1}{\partial #2 \partial #3}}
\newcommand{\N}{\mathbb{N}}
\newcommand{\Z}{\mathbb{Z}}
\newcommand{\C}{\mathbb{C}}
\newcommand{\Q}{\mathbb{Q}}
\newcommand{\R}{\mathbb{R}}
\newcommand{\K}{\mathbb{K}}
\newcommand{\sol}{\textbf{\emph{Soluci\'on: }}}
\newcommand{\dem}{\textbf{\emph{Demostraci\'on: }}}
\newcommand{\aux}[4]{\Large \textbf{Clase Auxiliar N#1: #2}}
\newcommand{\pauta}[4]{\Large \textbf{Pauta #1 N#2}}
\newcommand{\enc}[3]{\Large \textbf{#1}}
\newcommand{\norm}[1]{\lVert #1\rVert }
\newcommand{\vabs}[1]{\lvert #1\rvert}

\begin{document}

\fancyhead[L]{\itshape{Facultad de Ciencias F\'isicas y Matem\'aticas}}
\fancyhead[R]{\itshape{Universidad de Chile}}

\begin{minipage}{11.5 cm}
\begin{flushleft}
\hspace*{-0.6cm}\textbf{MA2001-4 Cálculo en Varias Variables}\\
\hspace*{-0.6cm}\textbf{Profesor:} Javier Ramírez G.\\
\hspace*{-0.6cm}\textbf{Auxiliar:} Alejandro Silva C.\\

\end{flushleft}
\end{minipage}

\begin{picture}(2,3)
    \put(370,-4){\includegraphics[scale=1.2]{fcfm2.pdf}}
\end{picture}

\begin{center}
	\LARGE \bf{Auxiliar \#11 }\\
\end{center}

\vspace{-1cm}
\begin{enumerate}\setlength{\itemsep}{0.4cm}	
\item[]

\item Considere la función $f:\mathbb{R}^2\rightarrow\mathbb{R}$ dada por:
\begin{align*}
    f(x,y)=\left\{
    \begin{matrix}
    \dfrac{xy(x^2-y^2)}{x^2+y^2}& \text{si } (x,y)\neq(0,0)\\
    0 & \text{si } (x,y)=(0,0)
    \end{matrix}\right. 
\end{align*}

\begin{enumerate}
    \item Calcule las derivadas direccionales en $(x,y)=(0,0)$
    \item Calcule el valor de $\dfrac{\partial^2 f}{\partial x\partial y}(0,0)$ y $\dfrac{\partial^2f}{\partial y\partial x}(0,0)$ sabiendo que para $(x,y)\neq(0,0)$
\begin{align*}
    \dfrac{\partial f}{\partial x}(x,y)=\dfrac{y(x^2-y^2)(x^2+y^2)+4x^2y^3}{(x^2+y^2)^2}, \qquad \dfrac{\partial f}{\partial y}(x,y)=-\dfrac{x(y^2-x^2)(y^2+x^2)+4y^2x^3}{(y^2+x^2)^2}
\end{align*}
¿Qué sucedió?
\end{enumerate}


\item 
\begin{enumerate}
    \item Sea $f:\mathbb{R}^2\rightarrow \mathbb{R}$ definida como:
    \[f(x,y)=ln(1+x^2+y^2)\]
    Determine el polinomio de Taylor de orden 2 para esta función en torno a $(x_0,y_0)=(1,1)$
    \item Sea $f(x,y)=x^3y^4e^{x+y}$. Encuentre el polinomio de grado 9 que mejor aproxima a la función $f(x,y)$ cerca de $(x,y)=(0,0)$
    \item Considere $f(x,y,z)=x^2y^4z+x^7yz+y^{10}cos(xy)-x^{10}y^{10}$. Encuentre su polinomio de orden 11 en torno a $(0,0,0)$
\end{enumerate}

\item Considere la función $f(x,y)=x^y$ definida para $x>0$ e $y\in\mathbb{R}$
\begin{enumerate}
    \item Encuentre el polinomio de Taylor de orden 2 ($P_2$) de $f$ en torno a $(1,0)$
    \item Pruebe que existe una constante $C$ tal que
    \[|f(1+h_1,h_2)-P_2(h_1,h_2)|\leq C (|h_1|+|h_2|)^3\]
    para $|h_1|+|h_2|\leq \frac{1}{2}$
\end{enumerate}
\item Sea $f:\mathbb{R}^n\rightarrow\mathbb{R}$ una función de clase $\mathcal{C}^2$ tal que para todo $x,h \in \mathbb{R}^n$ se tiene
\[|f(x+h)-f(x)-\nabla f(x)\cdot h|\leq \norm{h}^{7/3}\]
Pruebe que $H_f(x)=0$ en todo $x$ y deduzca que existen $a\in\mathbb{R}^n$, $b\in\mathbb{R}$ tales que $f(x)=a\cdot x+b$
\end{enumerate}
\end{document}