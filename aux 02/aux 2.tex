\documentclass[letterpaper,11pt]{article}
\oddsidemargin -1.0cm \textwidth 17.4cm

\usepackage[utf8]{inputenc}
\usepackage[activeacute,spanish]{babel}
\usepackage{amsfonts,setspace}
\usepackage{amsmath}
\usepackage{amssymb, amsmath, amsthm}
\usepackage{comment}
\usepackage{amssymb}
\usepackage{dsfont}
\usepackage{anysize}
\usepackage{multicol}
\usepackage{enumerate}
\usepackage{graphicx}
\usepackage[left=2cm,top=2cm,right=2cm, bottom=2cm]{geometry}
\setlength\headheight{2em} 
\usepackage{fancyhdr}
\pagestyle{fancy}
\fancyhf{}


\renewcommand{\labelenumi}{\normalsize\bfseries P\arabic{enumi}.}
\renewcommand{\labelenumii}{\normalsize\bfseries (\alph{enumii})}
\renewcommand{\labelenumiii}{\normalsize\bfseries \roman{enumiii})}


\DeclareMathOperator{\sen}{sen}
\DeclareMathOperator{\senh}{senh}
\DeclareMathOperator{\arcsen}{arcsen}
\DeclareMathOperator{\tg}{tg}
\DeclareMathOperator{\arctg}{arctg}
\DeclareMathOperator{\ctg}{ctg}
\DeclareMathOperator{\dom}{Dom}
\DeclareMathOperator{\sech}{sech}
\DeclareMathOperator{\rec}{Rec}
\DeclareMathOperator{\inte}{Int}
\DeclareMathOperator{\adh}{Adh}
\DeclareMathOperator{\fr}{Fr}
\DeclareMathOperator{\Ima}{Im}
\DeclareMathOperator{\dist}{dist}
\DeclareMathOperator{\argmin}{\text{argmín}}
\let\lim=\undefined\DeclareMathOperator*{\lim}{\text{lím}}
\let\max=\undefined\DeclareMathOperator*{\max}{\text{máx}}
\let\min=\undefined\DeclareMathOperator*{\min}{\text{mín}}
\let\inf=\undefined\DeclareMathOperator*{\inf}{\text{ínf}}


\newcommand{\pint}[2]{\left< #1,#2\right>}
\newcommand{\ssi}{\Longleftrightarrow}
\newcommand{\imp}{\Longrightarrow}
\newcommand{\pmi}{\Longleftarrow}
\newcommand{\ipartial}[2]{\dfrac{\partial #1}{\partial #2}}
\newcommand{\ider}[2]{\dfrac{d #1}{d #2}}
\newcommand{\iipartial}[2]{\dfrac{\partial^2 #1}{\partial #2^2}}
\newcommand{\iider}[2]{\dfrac{d^2 #1}{d #2^2}}
\newcommand{\ijpartial}[3]{\dfrac{\partial^2 #1}{\partial #2 \partial #3}}
\newcommand{\N}{\mathbb{N}}
\newcommand{\Z}{\mathbb{Z}}
\newcommand{\C}{\mathbb{C}}
\newcommand{\Q}{\mathbb{Q}}
\newcommand{\R}{\mathbb{R}}
\newcommand{\K}{\mathbb{K}}
\newcommand{\sol}{\textbf{\emph{Soluci\'on: }}}
\newcommand{\dem}{\textbf{\emph{Demostraci\'on: }}}
\newcommand{\aux}[4]{\Large \textbf{Clase Auxiliar N#1: #2}}
\newcommand{\pauta}[4]{\Large \textbf{Pauta #1 N#2}}
\newcommand{\enc}[3]{\Large \textbf{#1}}
\newcommand{\norm}[1]{\lVert #1\rVert }
\newcommand{\vabs}[1]{\lvert #1\rvert}

\begin{document}

\fancyhead[L]{\itshape{Facultad de Ciencias F\'isicas y Matem\'aticas}}
\fancyhead[R]{\itshape{Universidad de Chile}}

\begin{minipage}{11.5 cm}
\begin{flushleft}
\hspace*{-0.6cm}\textbf{MA2001-4 Cálculo en Varias Variables}\\
\hspace*{-0.6cm}\textbf{Profesor:} Javier Ramírez G.\\
\hspace*{-0.6cm}\textbf{Auxiliar:} Alejandro Silva C.\\

\end{flushleft}
\end{minipage}

\begin{picture}(2,3)
    \put(370,-4){\includegraphics[scale=1.2]{fcfm2.pdf}}
\end{picture}

\begin{center}
	\LARGE \bf{Auxiliar \#2}\\
\end{center}

\vspace{-1cm}
\begin{enumerate}\setlength{\itemsep}{0.4cm}	
\item[]


\item Sean $A, B$ subconjuntos de $\R^d$. Demuestre las siguientes propiedades:
\begin{enumerate}
    \item $\inte(A)\cup\inte(B) \subseteq \inte(A\cup B)$
    \item $\adh(A\cap B) \subseteq \adh(A)\cap\adh(B)$
    \item $\inte(A\cap B)=\inte(A)\cap \inte(B)$ 
    \item $\adh(A\cup B)=\adh(A)\cup\adh(B)$
    \item $\inte(A)=\phi$, si $A$ es numerable.
    
\end{enumerate}

\item Encuentre el interior, adherencia y frontera para los siguientes conjuntos. Determine si son abiertos/cerrados.

\begin{enumerate}
    \item $B$ = $\{  x\in \R^{n} : \norm{x}  > 2     \} \cup \R^n   $
    
    \item $ C =\{(x,y)\in\R^2:1\leq x^2 + y^2 \leq 2\}\cup\left\{ (x,y) \in \R^2: y=\frac{1}{x},x\in\{1,2,3,4,5\}\right\}$
\end{enumerate}

\item Sea la sucesión $(x_n, y_n)_{n\in \mathbb{N}} \subseteq \mathbb{R}^2$ definida por

$$\forall n \in \mathbb{N}, \quad (x_n, y_n) := \left( \frac{u_n(\cos(v_n)-1)}{u_n^2 + v_n^2}, \frac{2u_n^\alpha + v_n^4}{|u_n|+3|v_n|} \right);$$

con $(u_n)_{n\in \mathbb{N}}, (v_n)_{n\in \mathbb{N}}$ sucesiones reales tales que $\lim_{n \to \infty} u_n = \lim_{n \to \infty} v_n = 0$ y $\alpha \in \mathbb{R}$.
    
    
    \begin{enumerate}
    \item Muestre que $(x_n,y_n)_{n\in \mathbb{N}}$ converge cuando $\alpha = 2$ e indique su límite.
    \item ¿Qué pasa si $\alpha =1$?
    \end{enumerate}

\end{enumerate}	
\end{document}