\documentclass[letterpaper,11pt]{article}
\oddsidemargin -1.0cm \textwidth 17.4cm

\usepackage[utf8]{inputenc}
\usepackage[activeacute,spanish]{babel}
\usepackage{amsfonts,setspace}
\usepackage{amsmath}
\usepackage{amssymb, amsmath, amsthm}
\usepackage{comment}
\usepackage{amssymb}
\usepackage{dsfont}
\usepackage{anysize}
\usepackage{multicol}
\usepackage{enumerate}
\usepackage{graphicx}
\usepackage[left=2cm,top=2cm,right=2cm, bottom=2cm]{geometry}
\setlength\headheight{2em} 
\usepackage{fancyhdr}
\pagestyle{fancy}
\fancyhf{}


\renewcommand{\labelenumi}{\normalsize\bfseries P\arabic{enumi}.}
\renewcommand{\labelenumii}{\normalsize\bfseries (\alph{enumii})}
\renewcommand{\labelenumiii}{\normalsize\bfseries \roman{enumiii})}


\DeclareMathOperator{\sen}{sen}
\DeclareMathOperator{\senh}{senh}
\DeclareMathOperator{\arcsen}{arcsen}
\DeclareMathOperator{\tg}{tg}
\DeclareMathOperator{\arctg}{arctg}
\DeclareMathOperator{\ctg}{ctg}
\DeclareMathOperator{\dom}{Dom}
\DeclareMathOperator{\sech}{sech}
\DeclareMathOperator{\rec}{Rec}
\DeclareMathOperator{\inte}{Int}
\DeclareMathOperator{\adh}{Adh}
\DeclareMathOperator{\fr}{Fr}
\DeclareMathOperator{\Ima}{Im}
\DeclareMathOperator{\dist}{dist}
\DeclareMathOperator{\argmin}{\text{argmín}}
\let\lim=\undefined\DeclareMathOperator*{\lim}{\text{lím}}
\let\max=\undefined\DeclareMathOperator*{\max}{\text{máx}}
\let\min=\undefined\DeclareMathOperator*{\min}{\text{mín}}
\let\inf=\undefined\DeclareMathOperator*{\inf}{\text{ínf}}


\newcommand{\pint}[2]{\left< #1,#2\right>}
\newcommand{\ssi}{\Longleftrightarrow}
\newcommand{\conv}[2]{\xrightarrow[#1\to#2]{}}
\newcommand{\imp}{\Longrightarrow}
\newcommand{\pmi}{\Longleftarrow}
\newcommand{\ipartial}[2]{\dfrac{\partial #1}{\partial #2}}
\newcommand{\ider}[2]{\dfrac{d #1}{d #2}}
\newcommand{\iipartial}[2]{\dfrac{\partial^2 #1}{\partial #2^2}}
\newcommand{\iider}[2]{\dfrac{d^2 #1}{d #2^2}}
\newcommand{\ijpartial}[3]{\dfrac{\partial^2 #1}{\partial #2 \partial #3}}
\newcommand{\N}{\mathbb{N}}
\newcommand{\Z}{\mathbb{Z}}
\newcommand{\C}{\mathbb{C}}
\newcommand{\Q}{\mathbb{Q}}
\newcommand{\R}{\mathbb{R}}
\newcommand{\K}{\mathbb{K}}
\newcommand{\sol}{\textbf{\emph{Soluci\'on: }}}
\newcommand{\dem}{\textbf{\emph{Demostraci\'on: }}}
\newcommand{\aux}[4]{\Large \textbf{Clase Auxiliar N#1: #2}}
\newcommand{\pauta}[4]{\Large \textbf{Pauta #1 N#2}}
\newcommand{\enc}[3]{\Large \textbf{#1}}
\newcommand{\norm}[1]{\lVert #1\rVert }
\newcommand{\vabs}[1]{\lvert #1\rvert}

\begin{document}

\fancyhead[L]{\itshape{Facultad de Ciencias F\'isicas y Matem\'aticas}}
\fancyhead[R]{\itshape{Universidad de Chile}}

\begin{minipage}{11.5 cm}
\begin{flushleft}
\hspace*{-0.6cm}\textbf{MA2001-4 Cálculo en Varias Variables}\\
\hspace*{-0.6cm}\textbf{Profesor:} Javier Ramírez G.\\
\hspace*{-0.6cm}\textbf{Auxiliar:} Alejandro Silva C.\\

\end{flushleft}
\end{minipage}

\begin{picture}(2,3)
    \put(370,-4){\includegraphics[scale=1.2]{fcfm2.pdf}}
\end{picture}

\begin{center}
	\LARGE \bf{Auxiliar \#4}\\
\end{center}

\vspace{-1cm}
\begin{enumerate}\setlength{\itemsep}{0.4cm}	
\item[]

\item Sea $(x_k)_{k\in\N}\subseteq E$ (o $\R^d$ en su defecto) una sucesión, decimos que la \textit{serie} 
\[\sum_{k\in\N}x_k:=\lim_{N\to\infty}\sum_{k=1}^N x_k \]
es convergente si tal límite existe en $E$, es decir, si existe $x\in E$ tal que
\[\left\lVert x - \sum_{k\in\N}x_k \right\rVert \conv{N}{\infty}0.\]
    \begin{enumerate}
        \item Pruebe que si $\sum_{k\in\N}x_k$ es convergente, entonces $(x_k)_{k\in\N}$ también lo es, y más aún, converge a $0\in E$.
        \item Encuentre un ejemplo en el cual no se cumpla la recíproca (puede pensar en alguna sucesión real).
    \end{enumerate}

\item 
Se define la suma de conjuntos como:
\[A+B=\left\{a+b: a\in A  \wedge b\in B \right\}\]
    \begin{enumerate}
        \item Pruebe que si $A$ es un conjunto cerrado y $B$ es compacto, entonces $A+B$ es un conjunto cerrado.
        \item Pruebe que si $A$ y $B$ son compactos, entonces $A+B$ es un conjunto compacto.
        \item Para convencerse de que la compacidad de al menos uno de los conjuntos es necesaria para obtener lo deseado, considere los siguientes conjuntos en $\R$:
        $$A=\mathbb{N}\backslash \{0\},\ \ \ \ B=\left \{ -n+\frac{1}{n}\ :\
        n\in\mathbb{N}\right \}$$
        Demuestre que ambos son cerrados pero que $A+B$ no lo es.\\
    \end{enumerate}
\item Sea $(K_l)_{l\in\mathbb{N}}$ una familia de conjuntos compactos \textit{decrecientes} de $\mathbb{R}^n$, es decir: $K_{l+1}\subseteq K_{l}$ en cada $l$. Definamos $K=\bigcap_{l\geq0}K_l$
    \begin{enumerate}
        \item Pruebe que $K\neq\emptyset$
        \item Suponga que existe un abierto $A$ tal que $K\subseteq A$. Pruebe que existe $l$ tal que $K_l\subseteq A$ 
    \end{enumerate}
\end{enumerate}
\end{document}